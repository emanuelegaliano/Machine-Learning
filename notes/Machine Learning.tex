\documentclass[a4paper,11pt]{book}

% ===== Pacchetti essenziali =====
\usepackage[utf8]{inputenc}     % codifica sorgente (se usi pdfLaTeX)
\usepackage[T1]{fontenc}        % output font con accenti corretti
\usepackage[italian]{babel}     % sillabazione e testi automatici in italiano
\usepackage{lmodern}            % font Latin Modern
\usepackage{microtype}          % migliore giustificazione del testo
\usepackage{tikz}\usetikzlibrary{arrows.meta,positioning,decorations.pathreplacing}

\usepackage{listings}
\usepackage[T1]{fontenc}
\usepackage[scaled=0.9]{beramono} % facoltativo: font monospace più leggibile
\lstdefinestyle{py}{
  language=Python,
  basicstyle=\ttfamily\small,
  numbers=left,
  numbersep=8pt,
  frame=single,
  framerule=0.4pt,
  breaklines=true,
  tabsize=4,
  showstringspaces=false,
  captionpos=b
}

% ===== Matematica =====
\usepackage{amsmath,amssymb,amsthm}
\numberwithin{equation}{chapter}

% ===== Impaginazione =====
\usepackage{geometry}
% Margini comodi per tablet/stampa: non stretti, ma senza sprecare spazio
\geometry{a4paper, top=30mm, bottom=32mm, left=30mm, right=30mm}                                        

\usepackage{setspace}
% Interlinea: ben leggibile ma con buona densità di contenuto
\setstretch{1.15}

% Evita vedove/orfani (migliora la lettura su pagina singola)
\clubpenalty=10000
\widowpenalty=10000

% Profondità di numerazione / indice (come nello screenshot: 1.1.1)
\setcounter{secnumdepth}{3} % numerazione fino a \subsubsection
\setcounter{tocdepth}{2}    % TOC fino a \subsection (metti 3 se vuoi anche le subsubsection)

% ===== Intestazioni e piè di pagina =====
\usepackage{fancyhdr}
\pagestyle{fancy}
\fancyhf{} % pulisci
\fancyhead[LE,RO]{\thepage}
\fancyhead[LO]{\nouppercase{\rightmark}}
\fancyhead[RE]{\nouppercase{\leftmark}}
\setlength{\headheight}{15pt}
\addtolength{\topmargin}{-2pt}

% ===== Tabelle, grafica,    link =====
\usepackage{booktabs}
\usepackage{graphicx}
\usepackage[hidelinks]{hyperref}

% ===== Liste: leggibili, non compattate eccessivamente =====
\usepackage{enumitem}
% itemize / enumerate: spazi moderati per mantenere i dettagli visivi
\setlist[itemize]{topsep=4pt, partopsep=0pt, itemsep=3pt, parsep=2pt}
\setlist[enumerate]{topsep=4pt, partopsep=0pt, itemsep=3pt, parsep=2pt}
% description: rientro e allineamento gradevole; etichetta in grassetto
\setlist[description]{font=\normalfont\bfseries, labelsep=0.6em, leftmargin=2em}

% (Facoltativo) Se vuoi che gli elenchi siano più “a colonna” in certi punti:
% \begin{description}[widest=Etichetta-più-lunga,leftmargin=!,labelsep=0.6em]
%   \item[Etichetta] Testo...
% \end{description}

% ===== Metadati =====
\title{\Huge \bfseries Machine Learning}
\author{\Large Emanuele Galiano \\
\Large Damiano Trovato}
\date{Anno Accademico 2025/2026}

\begin{document}

\frontmatter
\maketitle
\tableofcontents

\mainmatter
\chapter{Introduzione al Machine Learning}

\section{Definizione di Machine Learning (Arthur Samuel - 1959)}


Nel 1959, Arthur Samuel fornisce una \textbf{definizione di machine learning}: il machine learning è il campo di studi che abilita i computer ad \textbf{imparare}, \textbf{senza essere esplicitamente programmati}.

Il paradigma alla base è fortemente diverso da quello tipico: se un algoritmo classico è \textbf{programmato ad hoc }per risolvere un task, e si comporta in maniera prettamente \textbf{deterministica}, un algoritmo di machine learning può \textbf{imparare a risolvere problemi} associati a vasti set di dati (classificare elementi, riconoscerli, trovare correlazioni). Questo permette di spostare il nostro focus non più sullo sviluppo di tutti gli step necessari affinchè un algoritmo risolvi \textbf{un problema specifico}, ma sulla creazione di un modello che riesca ad apprendere come risolvere \textbf{un insieme di problemi simili}.

\section{Perchè abbiamo bisogno del Machine Learning}

L'approccio utilizzato finora per risolvere i problemi è quello di:

\begin{itemize}
\item Trovare una logica per risolvere il problema.
\item Scrivere un programma.
\item Suddividerlo in pezzi più piccoli (funzioni).
\item Automatizzare l'approccio.
\end{itemize}

Questo funziona per problemi di natura fortemente univoca, che sappiamo come risolvere, ad esempio:

\begin{itemize}
    \item Computare l'area di un poligono.
    \item Risolvere equazioni differenziali.
\end{itemize}

Nel caso del poligono, supponendo di voler calcolare l'area di un rombo i dati presi in input sarebbero dati dalla coppia $(x_1,x_2)$, contenente le lunghezze della diagonale principale e secondaria.
Questi dati,passati ad un algoritmo, permettono di calcolarne l'area $\frac{x1*x2}{2}$ e generarne un output

$$
\text{Dati} \to \text{Programma che risolve un task} \to \text{Output}
$$

Alcuni problemi tuttavia presentano un alto grado di \textbf{incertezza}, che li rende più difficili da affrontare. Non poter fare assunzioni sui dati in input, e non conoscere tutti i possibili task, rende impossibile l'utilizzo di algoritmi standard per compiti del tipo:

\begin{itemize}
\item Classificazione di email spam e non spam
\item Object detection
\end{itemize}

Il machine learning rappresenta la soluzione ideale a problemi di questo tipo, proponendo una nuova pipeline:

$$
\text{Dati + Output Atteso} \to \text{Machine Learning} \to \text{Soluzioni su nuovi dati}
$$

\subsection{Esempio delle email spam e non-spam}

Vogliamo creare un algoritmo di machine learning in grado di determinare se una mail è spam o meno. Il nostro obiettivo è quindi classificare ciascuna di queste come \textbf{spam}, o \textbf{ham}\footnote{Email legittima, non spam.}.

\begin{itemize}
    \item \verb|Compra prodotto a 10$! Oferta imperdibile! |$\rightarrow$ Spam
    \item \verb|Ciao Giovanni, come stai?| $\rightarrow$ Ham
\end{itemize}

\section{Machine Learning Algorithm (Tom Mitchel - 1998)}

Un algoritmo \textbf{apprende dall'esperienza} $E$ rispetto a una certa classe di \textbf{Task} $T$ e a una misura di \textbf{performance} $P$. Se la sua \textbf{performance} nel compito $T$, misurata tramite $P$, migliora con l’\textbf{esperienza }$E$, allora quel modello ha appreso con successo.

\section{Definizione di Task}

Rappresenta il problema che deve essere risolto. Nell'esempio di determinare se una mail è spam o meno, il task è quello di \textbf{predire} l'etichetta ($Y=$"spam" oppure $Y=$"ham"), ed è strettamente legata al modello, che rappresentiamo come funzione parametrizzata, indicata con $h_\theta$.

\section{Definizione di Esperienza}

Rappresentano i dati, ovvero i valori assunti dalle \textbf{random variables}, nell'esempio $X$ è il contenuto della mail ed $Y$ l'etichetta. La coppia di valori:

$$
  {\{(X=x_i, Y=y_i)\}}_{i=1}^N
$$

\noindent
Rappresenta l'esperienza. Generalmente vista come una collezione di elementi chiamati \textbf{esempi}.

\section{Definizione di Performance}

Funzione $P$ che \textbf{valuta quanto bene } il modello è in grado di \textbf{risolvere un certo task} $T$.
Supponiamo che il nostro algoritmo abbia previsto un insieme di etichette per un dato numero di email che indichiamo con:

$$ \{\hat{y_i}\} $$


Dove il simbolo 'hat' indica che il dato non è stato osservato ma \textbf{previsto}. L'insieme delle etichette corrette è invece dato da

$$ \{{y_i}\} $$


Per valutare la qualità del nostro metodo, dovremmo confrontare i due insiemi di previsioni utilizzando una \textbf{misura di performance:}

$$ P(\{{y_i}\},\{\hat{y_i}\} ) $$

\noindent
Questa funzione restituisce un valore reale appartenente al range [0,1].

\begin{itemize}
\item Un \textbf{valore elevato} indica che le previsioni sono accurate
\item Un \textbf{valore basso} indica che le previsioni non sono accurate.
\end{itemize}

\noindent
Indichiamo con il termine \textbf{misura di errore} il valore: $1 - P$. Per risolvere problemi di machine learning ci affidiamo a modelli statistici che dipendono dal task.

\section{Esempio completo}

Siano:

\begin{itemize}
\item $x^{(1)}$: Il testo dell'email 1: "Compra prodotto a 10\$! Oferta imperdibile!"
\item $x^{(2)}$: Il testo dell'email 2: "Ciao Giovanni, come stai?"

\item $y^{(1)}$: L'etichetta \textbf{spam}
\item $y^{(2)}$: L'etichetta \textbf{ham}
\item $h_\theta$: Il modello
\end{itemize}

\noindent
Allora

$$ h_\theta(x^{(1)}) = \hat{y}^{(1)} $$

\noindent
e

$$ h_\theta(x^{(2)}) = \hat{y}^{(2)} $$

\section{Task, Esempi ed Etichette}

Un esempio è generalmente espresso come una raccolta di valori che sono stati misurati quantitativamente da un evento osservato. Un esempio è generato da un vettore:

$$ x \in \mathbb{R}^{d} $$

\noindent
Scritto anche come:

$$ x = (x_1, x_2, ..., x_d)$$

I valori del vettore $x$ sono detti \textbf{features}, in quanto rappresentano \textbf{proprietà specifiche} degli esempi in input. Se la dimensionalità di $x$ è 10, diremo che ha 10 features Nella maggior parte dei casi, ogni esempio $x$ è anche abbinato a un output desiderato $y$. Tali output desiderati, sono anche chiamati \textbf{etichette}. Un'attività può quindi essere definita come un certo modo di elaborare un esempio di input per ottenere un output.

Torniamo al nostro esempio: determinare se un'e-mail è spam o ham. In questo caso, l'input è l'email, le features possono essere caratteristiche dell'email, come il numero di errori ortografici o la presenza di alcune parole chiave, mentre l'output atteso è l'etichetta (spam o ham).

\section{Estrazione delle features}

Per gestire le email, dobbiamo prima trasformarle in un'\textbf{entità quantificabile}. Questo di solito viene fatto \textbf{identificando alcune caratteristiche} dei dati che sono \textbf{rilevanti per il compito dato} (numero di errori ortografici o la presenza di alcune parole chiave). In pratica, stiamo cercando una funzione $f$ che trasformi l'entità dalla sua forma originale a una forma di destinazione, che è buona per risolvere un compito specifico:

$$
x \rightsquigarrow f(x) \rightsquigarrow \overline{x}
$$

Dove $x$ è il dato grezzo di input (ad esempio, il messaggio di posta elettronica completo), $f$ è la funzione di trasformazione e $\overline{x}$\footnote{Su questa notazione: da ora in poi, quando ci riferiremo all'output della trasformazione, non useremo più $\overline{x}$, ma direttamente $x$, dando per scontato il passaggio di rappresentazione $f(x)$.} è l'output della trasformazione, che sarà l'input dell'algoritmo di apprendimento automatico.

La funzione $f$ è chiamata \emph{rappresentazione}. L'output della trasformazione $x$ è anche chiamato rappresentazione.
Poiché rappresentando i dati otteniamo un vettore di funzionalità, il processo di rappresentazione dei dati è talvolta chiamato \textbf{features extraction}. Non ci sono «rappresentazioni universali», ma solo rappresentazioni che servono a qualche compito.

\noindent
Le rappresentazioni sono di 2 tipi:

\begin{itemize}
\item Create a mano
\item Apprese
\end{itemize}

\noindent
L'estrazione delle features \textbf{mette in luce caratteristiche salienti} trascurandone altre.

\section{Features}

Generalmente, l'output di una funzione di rappresentazione è nella forma:

$$
x = (x_1, x_2, \dots, x_d), \quad x \in \mathbb{R}^d
$$

\noindent
Questo, è composto da un insieme di features . \textbf{Una feature è la specifica di un attributo}.
Si tratta di una misura che rappresenta \textbf{aspetti dei dati }che è utile \textbf{evidenziare per risolvere il problema considerato}. Ad esempio, il colore può essere un attributo. "Il colore è blu" è una funzionalità estratta da un esempio.

\noindent
Le caratteristiche possono essere di due tipi principali:

\begin{description}
    \item[Categoriche:] un numero finito di valori discreti. Questi possono essere:
    \begin{itemize}
        \item \textbf{Nominali:} a indicare che non esiste \textbf{alcun ordinamento} tra i valori, ad esempio cognomi e colori.
        \item \textbf{Ordinali:} a indicare che esiste un \textbf{ordinamento rilevante}, ad esempio in un attributo che assume i valori basso, medio o alto.
    \end{itemize}
    \item[Continue:] comunemente, \textbf{sottoinsieme di numeri reali}, dove c'è una differenza misurabile tra i valori possibili. I numeri interi sono solitamente trattati come continui nei problemi pratici.
\end{description}

\section{Esempio delle Email Spam e Non Spam}

Consideriamo il nostro esempio in cui vogliamo distinguere le e-mail spam da quelle non spam.
L'input del processo sono i messaggi di posta elettronica, quindi dobbiamo trasformarli in vettori di features:

$$ x = (x_1, x_2, ..., x_n)$$

\noindent
con un processo di \emph{features extraction}.

Naturalmente, ci aspettiamo che le funzionalità estratte siano\textbf{ utili per risolvere il nostro compito} di determinare se un'e-mail è spam o ham. Possiamo notare che le e-mail di spam spesso includono errori ortografici e parole come "Acquista", "occasione" e "10\$". Quindi, potremmo decidere di rappresentare ogni messaggio di posta elettronica con due numeri:

\begin{itemize}
    \item Il conteggio degli errori ortografici.
    \item Il numero di volte in cui alcune parole o pattern specifici appaiono nel testo.
\end{itemize}

Una volta che i messaggi di input sono stati convertiti in \textbf{vettori di funzionalità}, possono essere visti come \textbf{vettori nello spazio} $\mathbb{R}^2$.

\begin{figure}[htbp]
    \centering
    \includegraphics[width=\textwidth]{images/featureExtraction.png}
    \caption{Estrazione delle feature: le e-mail vengono trasformate in vettori 2D, dove $x = \text{errori ortografici e } y = \text{pattern ripetibili}$, e proiettate nello spazio delle feature, , dove la separazione tra spam (arancione) e non spam (verde) risulta evidente.}
    \label{fig:featureExtraction}
\end{figure}

\section{Tipologie di Task}

Le attività possono essere di diversi tipi. Di seguito, discuteremo due compiti principali:

\begin{itemize}
\item \textbf{Classificazione}
\item \textbf{Regressione}
\end{itemize}

Assumeremo che ogni algoritmo di apprendimento automatico prenda come input esempi che sono già stati rappresentati con una funzione di rappresentazione adeguata.

\subsection{Classificazione}

In questo tipo di attività, alla macchina viene chiesto di specificare a quale di un insieme predefinito di categorie $K$ appartiene l'input.

\noindent
Esempi di questo compito sono:

\begin{itemize}
\item Classificare i post di Facebook come riguardanti la politica o qualcos'altro (classificazione politica vs non politica).
\item Rilevamento delle e-mail di spam (classificazione dello spam vs legittima delle e-mail).
\item Riconoscimento dell'oggetto raffigurato in un'immagine tra 1000 oggetti diversi (riconoscimento dell'oggetto).
\end{itemize}

\noindent
L'algoritmo di apprendimento è solitamente fornito con un insieme di esempi:

$$ \{x^{(1)}, x^{(2)}, ..., x^{(n)}\} \text{ dove: } x^{(j)} \in \mathbb{R}^{N} \forall j$$

\noindent
e un insieme di etichette corrispondenti

$$ \{y^{(1)}, y^{(2)}, ..., y^{(n)}\} \text{ dove: } y^{(j)} \in \{1,..,k\}\forall j$$

\noindent
che specificano a quale delle categorie K appartiene ogni esempio.

Ad esempio, se $y^{(j)} = 3$, allora $x^{(j)} $ appartiene alla classe "3".

Nel caso della classificazione binaria (ad esempio, spam vs non spam), $y^{(j)} \in \{0,1\} $. Per risolvere questo compito, l'algoritmo di apprendimento automatico assume la forma di una funzione:

$$ h_\theta: \mathbb{R}^{N} \rightarrow \{1, ... ,K\} $$

\noindent
tale che:

$$y^{(j)} = h_\theta(x^{(j)})$$

\noindent
Esempio:

\begin{itemize}
    \item \textbf{Classification Task:} data un'e-mail, classificarla come spam o non spam.
    \item \textbf{Input:} esempi n-dimensionali $ x = (x_1, x_2, ..., x_n)$ contenenti le caratteristiche dell'email, come il numero di errori ortografici e l'occorrenza di parole specifiche.
    \item \textbf{Output:} etichette $y \in \{0,1\}$ che indicano se l'e-mail è legittima o spam.
\end{itemize}

Alcuni algoritmi di classificazione \textbf{non prevedono un output discreto}, ma un vettore di \textbf{probabilità}, contenente la probabilità relativa a ciascuna delle etichette possibili. In questo caso, puntiamo ad avere la probabilità massima nello slot del vettore relativo all'etichetta vera. 

\subsection{Regressione}

In questo tipo di compito, al programma del computer viene chiesto di \textbf{prevedere un valore numerico dato un input}, tipo:
\begin{itemize}
    \item Prevedere il prezzo delle case date alcune caratteristiche come la città, l'età, la zona, ecc.
    \item Prevedere il valore futuro delle azioni di una società dai valori di altre società o da altre statistiche sul mercato (previsione del mercato azionario).
    \item Conta il numero di auto presenti in un'immagine.
\end{itemize}

\noindent
Analogamente alla classificazione, l'algoritmo viene fornito con esempi di training $x \in \mathbb{R}^{N}$ e con gli output desiderati $y \in \mathbb{R}$. L'algoritmo di apprendimento automatico assume la forma di una funzione $ h_\theta: \mathbb{R}^{N} \rightarrow \mathbb{R}$ tale che $y^{(j)} = h_\theta(x^{(j)})$.

\noindent
Esempio:

\begin{itemize}
    \item \textbf{Regression task:} Predire il prezzo di una casa in base ai suoi metri quadrati.
    \item \textbf{Input:} Dimensione della casa $x$ (valore scalare)
    \item \textbf{Output:} Prezzo $y$.
\end{itemize}

\noindent
Sono algoritmi ottimi per trovare relazioni tra i dati ed effettuare predizioni.

\begin{figure}[htp]
    \centering
    \includegraphics[width=\textwidth]{images/regression.png}
    \caption{Relazione tra dimensione dell’immobile $(x_11^{(i)}$, in mq) e prezzo $(y^{(i)}$, in migliaia di \$): i punti rossi sono i dati osservati, la linea blu rappresenta un modello di regressione lineare che non approssima bene l’andamento non lineare.}
    \label{fig:regressione}
\end{figure}

\newpage
\section{Supervised Learning e Unsupervised Learning}

Gli approcci di Machine Learning possono essere approssimativamente divisi in \textbf{supervised} e \textbf{unsupervised learning}.

\paragraph{Supervised Learning:} L'algoritmo viene addestrato su un insieme di esempi di input e \textbf{output desiderati}. L'obiettivo è allenare un modello a mappare gli input agli output corretti. Il punto chiave qui è il conoscere gli output desiderati: i dati del nostro dataset dovranno quindi essere preventivamente \textbf{etichettati}.

\paragraph{Unsupervised Learning:} L'algoritmo viene addestrato solo su esempi di input, senza output desiderati. L'obiettivo è trovate struttura, pattern e associazioni nei dati, spesso molto eterogenei.

$$ 
\{x^{(1)}, x^{(2)}, ..., x^{(n)}\} \qquad \text{ dove: } x^{(j)} \in \mathbb{R}^{N} \forall j
$$

Questi tipi di compiti mirano generalmente a \textbf{modellare la struttura dei dati}. Un esempio di unsupervised learning è il \textbf{clustering}, in cui non viene fornita alcuna informazione aggiuntiva oltre agli esempi.

Gli approcci supervised sono generalmente più facili da gestire, ma richiedono la presenza di \textbf{labels}. Ottenere labels è spesso un problema costoso in termini di tempo, poiché richiede che le persone annotino manualmente i dati. Ad esempio, se dobbiamo costruire un spam-detector utilizzando un approccio supervised, è necessario che qualcuno etichetti manualmente diverse email come ‘spam’ o ‘non-spam’.

\newpage

\section{Reinforcement Learning}

Alcuni autori fanno riferimento anche a una terza classe di algoritmi di Machine Learning: il \textbf{Reinforcement Learning}.

Il Reinforcement Learning mira a \textbf{scoprire la soluzione a un problema} attraverso il metodo \emph{trial and error}, piuttosto che tramite istruzioni esplicite su come risolvere il compito. Questo avviene permettendo all'algoritmo di \textbf{interagire con un environment} e ricevere \textbf{positive rewards} quando compie azioni che portano a un buon risultato (rispetto al problema da risolvere) e \textbf{negative rewards} quando compie azioni che portano a un risultato negativo.

L'obiettivo degli algoritmi di Reinforcement Learning è apprendere una policy $\pi$, che possa essere utilizzata per \textbf{determinare quale azione a intraprendere} quando si acquisisce un'\textbf{osservazione} del mondo $o$. Questo processo \textbf{ricorda il modo naturale in cui gli animali imparano a risolvere problemi}. Ad esempio, si può pensare a un topo che deve trovare l'uscita da un labirinto (immagine \ref{fig:reinforcementLearning}).

\begin{figure}[htp]
    \centering
    \includegraphics[width=\textwidth]{images/reinforcementLearning.png}
    \caption{Reinforcement Learning nel labirinto: l’agente sceglie tra azioni U/D/L/R e apprende una policy \(\pi(o)=a\) che massimizza la ricompensa, raggiungendo il formaggio (positiva) ed evitando il gatto (negativa).}
\label{fig:reinforcementLearning}
\end{figure}

\section{Misura di Performance (P)}

Per valutare le capacità di un algoritmo di Machine Learning nel risolvere un determinato compito, è necessaria una misura quantitativa delle sue prestazioni. Solitamente, questa \textbf{performance measure} \( P \) è \textbf{specifica per il task} \( T \) che il sistema sta eseguendo.

Per compiti come la classification, spesso si misura la performance utilizzando l'\textbf{accuracy}, ovvero la percentuale di esempi classificati correttamente dal modello. Nel caso della regression, invece, si possono usare altre metriche come il mean squared error.

\noindent
Le misure di performance sono utilizzate per due motivi principali:

\begin{itemize}
\item Capire quando un algoritmo di Machine Learning sta migliorando in un determinato compito.
\item Valutare la performance dell'algoritmo una volta finalizzato.
\end{itemize}

\noindent
Una performance measure può anche essere vista in termini di error. Ad esempio, l'\textbf{accuracy} corrisponde a un error rate (la percentuale di esempi classificati in modo errato), calcolato come \( 1 - accuracy \).

\subsection{Esempio}

Un spam detector analizza cinque email. Le prime tre sono spam, le ultime due non lo sono. L'algoritmo classifica come spam le prime due email e come non spam le ultime tre. In questo caso, la prima e le ultime due classificazioni sono corrette, mentre la terza è errata. La accuracy si calcola come la percentuale di esempi classificati correttamente:

$$
\frac{4}{5} = 0.8 \quad \text{ovvero} \quad 80\%
$$

\section{Experience (E)}

Un algoritmo di Machine Learning apprende dall'\textbf{experience} per migliorare una performance measure su un determinato task.

\noindent
L'\textbf{experience} è costituita da una raccolta di esempi

\[
 x^{(i)}
\]

\noindent
(noti anche come data points, poiché possono essere mappati in uno spazio multi-dimensionale tramite una funzione di rappresentazione), eventualmente accompagnati dalle relative labels 

\[ y^{(i)} \]

\noindent
(a seconda del task considerato).

\noindent
Esistono due principali tipi di algoritmi di Machine Learning:

\begin{itemize}
\item Supervised approaches (quando abbiamo le paired labels, ad esempio nella classification e nella \textbf{regression}).
\item Unsupervised approaches (quando non abbiamo paired labels, come nel \textbf{clustering}).
\end{itemize}

\noindent
L'\textbf{experience} assume forme diverse a seconda del tipo di approccio di Machine Learning utilizzato.

\section{Dataset (D)}

Le performance measures vengono generalmente calcolate rispetto a un insieme di esempi, piuttosto che su singoli esempi. Un insieme di esempi (eventualmente con labels) è chiamato \textbf{dataset}. I datasets sono generalmente omogenei, nel senso che i dati contenuti al loro interno hanno un formato simile. Ad esempio:

\begin{itemize}
    \item Nel Fisher’s Iris dataset, tutti gli esempi hanno 4 features e una label corrispondente a una delle tre classi.
    \item In un dataset di immagini di food, ogni immagine è associata a una class che indica il piatto specifico.
\end{itemize}

\section{Design Matrix}

Un modo comune per rappresentare un dataset è utilizzare una design matrix. Poiché ogni esempio è una collezione di $n$ features, un dataset di $m$ elementi può essere rappresentato tramite una matrice

$$
X \in \mathbb{R}^{m \times n}, \ m, n \in \mathbb{N}
$$

\noindent
di dimensione $m \times n$.

\begin{itemize}
    \item Ogni riga della design matrix rappresenta un esempio.
    \item Ogni colonna rappresenta una delle features.
\end{itemize}

\noindent
Nel caso del supervised learning, si considera spesso anche un'altra matrice

$$ Y \in \mathbb{A}^{m \times k}, k \in \mathbb{N}$$

\noindent
dove \( k \) è la dimensionalità degli output desiderati.

\noindent
Ad esempio, nel caso della classification,

\[ \mathbb{A} = \{1, \ldots, M\} \]

\noindent
dove \( M \) è il numero di classi e k è spesso uguale a 1.

\begin{figure}[htbp]
    \centering
    \includegraphics[width=\textwidth]{images/designMatrix.jpg}
    \caption{Design Matrix: rappresentazione di un dataset con m esempi e n features. Ogni riga corrisponde a un esempio, ogni colonna a una feature.}
    \label{fig:designMatrix}
\end{figure}

\subsection{Esempio}

Supponiamo di avere un dataset composto da 1000 email, alcune classificate come spam e altre come not spam. Assumiamo che ogni email sia rappresentata da due features, come discusso nei precedenti esempi.

\noindent
La design matrix che rappresenta il dataset è una matrice

$$ X \in \mathbb{R}^{1000 \times 2} $$

\begin{itemize}
\item Ogni elemento della matrice rappresenta una delle features di un esempio nel dataset.
\item Ad esempio, $ {X}_{i,1} $ indica il numero di \textbf{errori} ortografici nell’$i$-\textbf{esima email}, mentre $ {X}_{j,2} $ rappresenta il numero di \textbf{occorrenze} di parole chiave nell’$j$-\textbf{esima email}, e così via.
\end{itemize}

\noindent
Le labels sono contenute in un vettore

$$ \mathbf{Y} \in \{0,1\}^{1000} $$
dove $ Y_i $ rappresenta la label associata all’$i$-\textbf{esimo esempio} (ad esempio, 0 = not spam, \textbf{1 = spam}).

\section{Learning}

Un algoritmo di Machine Learning \textbf{utilizza un dataset di esempi} per \textbf{migliorare la sua performance} in un determinato \textbf{task}. Il processo di miglioramento della performance dell'algoritmo è chiamato \textbf{learning} o \textbf{training}.

\subsection{In cosa consiste il training?}

\begin{itemize}
\item Un algoritmo di Machine Learning ha alcuni parametri chiamati parameters, che possono essere regolati per modificarne il comportamento. Questi parametri sono legati a un model (una funzione matematica) utilizzata per risolvere il task.
\item Un algoritmo chiamato training procedure utilizza gli esempi forniti per trovare i valori ottimali per questi parameters.
\item Alcuni parametri non possono essere regolati automaticamente dal training. Questi sono detti hyperparameters e devono essere ottimizzati al di fuori della training procedure, spesso attraverso un metodo trial and error.
\end{itemize}

\subsection{Esempio}

Consideriamo un semplice spam detector che classifica le email come spam o non-spam in base al numero di errori ortografici. 

\noindent
L'algoritmo può essere scritto come segue:


\begin{lstlisting}[style=py,caption={Soglia sul numero di errori ortografici, chiaramente un approccio naive},label={lst:spam-threshold}]
def classify(x):
    if x > a:
        return 1  # Spam
    else:
        return 0  # Non-spam
\end{lstlisting}

\noindent
L'algoritmo dipende da un singolo parametro $a$. La domanda è: quale valore dovremmo assegnare ad $a$? La \emph{training procedure} permette di trovare un valore ottimo presumibilmente adatto per $a$.
\noindent
Una semplice training procedure consisterebbe nel provare diversi valori per $a$ e registrare le performance dell'algoritmo per ciascun valore di $a$. Alla fine, possiamo scegliere il valore di $a$ che massimizza la performance measure P.

\chapter{I primi modelli di Machine Learning}

\section{Neurone computazionale - modello di McCulloch-Pitt}
L'idea iniziale era quella di replicare la struttura di un neurone biologico con un modello matematico, in modo da poter simulare il funzionamento del cervello umano. 

\paragraph{Neurone biologico.} Un neurone biologico è costituito da:
\begin{itemize}
	\item Dendriti: ricevono segnali da altri neuroni.
	\item Corpo cellulare: elabora i segnali ricevuti.
	\item Assone: trasmette il segnale elaborato ad altri neuroni.
\end{itemize}

\noindent
Il primo modello di neurone computazionale fu proposto da McCulloch e Pitts nel 1943. Si basa su tre componenti principali: 

\begin{itemize}
	\item Degli input $x_1, x_2, \dots, x_d$ binari, che possono essere \textbf{eccitatori }e \textbf{inibitori}. 
	\item Una threshold $v$, una soglia
	\item Un output binario.
\end{itemize}

\begin{figure}[tbph]
	\centering
	\includegraphics[width=\linewidth]{./images/neurone_computazionale.pdf}
	\caption{Modello di McCulloch Pitt e due computazioni possibili: AND logico e OR logico}
	\label{fig:neuronecomputazionale}
\end{figure}

Supponiamo che $x_1, \dots, x_j$ siano eccitatori, $x_{j+1}, \dots, x_d$ sono inibitori. Se $j \geq 1$ e almeno un inibitore è $=1$, allora il neurone ritorna 0. Un inibitore è sufficiente per bloccare l'output. Altrimenti, si calcola $z = x_1 + \dots + x_j = x_1 + \dots + x_d$, ossia la somma di tutti gli input\footnote{posso considerare anche gli input degli inibitori in quanto $=0$.}. Se la somma è $\geq v$, allora l'output sarà 1, altrimenti 0.

Questo modello è tale da poter computare un $AND$ ($v=n$ e $n$ input eccitatori), un $OR$ ($v=1$ e $n$ input eccitatori, vedere figura \ref{fig:neuronecomputazionale}), ma non uno $XOR$!

\subsection{Limitazioni del modello di McCulloch-Pitt}
\begin{itemize}
	\item Non esiste un modo automatico di fare training.
	\item Gli input sono binari.
	\item Gli input hanno tutti lo stesso peso.
	\item Tutte le funzioni computabili sono linearmente separabili.
\end{itemize}

\section{Percettrone - Rosenblatt}

Il successore del modello di McCulloch-Pitt è il \textbf{percettrone}. Questo modello risolve il problema dei pesi, assegnando ad ogni ingresso un peso differente, e introduce una procedura di apprendimento automatico per stabilire i pesi in maniera opportuna.
\begin{figure}[tbph]
	\centering
	\includegraphics[width=\linewidth]{./images/percettrone.pdf}
	\caption{Modello di Rosenblatt, noto come Percettrone}
	\label{fig:percettrone}
\end{figure}

\noindent
Le componenti sono:
\begin{itemize}
	\item Features $x_1, x_2, \dots, x_d$ normalizzate in $[0, 1]$. Ciascuno di questi input ha associato un peso $\vartheta_1, \dots, \vartheta_d$.
	\item Una threshold $\vartheta_0$, una soglia.
	\item Un output binario.
	\item Una \textbf{procedura di learning automatico} per stabilire i parametri (il peso di ciascun input).
\end{itemize}

\noindent
Il comportamento del percettrone sarà il seguente:

$$
f(x_1, \dots, x_d) = 
\begin{cases}
	1 \qquad \displaystyle\sum_{j=1} x_j v_j \geq \vartheta_0\\
	0 \qquad \text{altrimenti}
\end{cases}
$$

Per semplificare la computazione, verrà aggiunta una feature $x_0 = 1$ con peso $\vartheta_0$, uguale alla threshold. Questo ci aiuterà a scrivere la funzione con la seguente notazione:
$$
f(x_1, \dots, x_d) = 
\begin{cases}
	1 \qquad \displaystyle\sum_{j=1} x_j v_j \geq 0\\
	0 \qquad \text{altrimenti}
\end{cases}
$$

\noindent
E poi come prodotto matriciale:
$$
f(x) = [\vartheta^T \overline{x} > 0]
$$

\noindent
dove $\vartheta^T$ è il vettore dei pesi. Supponiamo ora di avere 2 parametri, andremo a ottenere:
$$
\vartheta_0 + x_1\vartheta_1 + x_2\vartheta_2 \geq 0
$$

e con dei semplici passaggi, possiamo tracciare una retta nello spazio, chiamata \textbf{decision boundary}, che dividerà lo spazio in due parti, quella per cui $f(x_1, \dots, x_d) = 1$, e quella per cui $f(x_1, \dots, x_d) = 0$
$$
x_2 = x_1\frac{\vartheta_1}{\vartheta_2} + \frac{\vartheta_0}{\vartheta_2}
$$

\begin{figure}[htbp]
	\centering
	\includegraphics[width=0.7\textwidth]{./images/decision_boundary.png}
	\caption{Decision boundary del percettrone in 2D. I punti arancioni sono classificati come 1, quelli blu come 0 e la retta al centro è la \emph{decision boundary}.}
	\label{fig:decisionboundary}
\end{figure}

Questa retta, in un task di classificazione, suddivide lo spazio in due classi. In uno spazio 2D è una retta, in uno spazio 3D un piano. Lo spazio dovrà essere \textbf{linearmente separabile}.

\subsection{Processo generale di training del percettrone}
Descriviamo la procedura di training nel seguente modo:

\begin{enumerate}
	\item Inizializzazione casuale dei pesi $\vartheta_1, \dots, \vartheta_d \in \mathbb{R}$.
	\item Computa $\forall x^{(i)}$ il valore di $\hat{y}^{(i)}$.
	\item Confronta $\hat{y}^{(i)}$ con $y^{(i)}$
	\begin{itemize}
		\item Se  $\hat{y}^{(i)} = y^{(i)}$: non fare nulla.
		\item Se $\hat{y}^{(i)} \neq y^{(i)}$: vanno aggiornati i pesi. Analizziamo come modificare i pesi, in funzione dei risultati.
	\end{itemize}
	
	\item Aggiornamento dei pesi:
	
	\begin{center}
		\begin{tabular}{|c|c|c|}
			\hline
			$\hat{y}^{(i)}$ & $y^{(i)}$ & Cosa fare\\
			\hline
			0 & 0 & ok!\\
			0 & 1 & riduci i pesi.\\
			1 & 0 & aumenta i pesi.\\
			1 & 1 & ok!\\
			\hline
		\end{tabular}
	\end{center}
	
	Questo sposterà la retta in maniera opportuna, convergendo ad un risultato opportuno per il dataset di training.
\end{enumerate}

Questa descrizione generale riflette a grandi linee il processo, tuttavia è opportuno conoscere la procedura dal punto di vista matematico, che riflette poi l'implementazione effettiva.

\subsection{Implementazione del training}
L'aggiornamento dei pesi avviene nel seguente modo:
$$
\underbrace{\hat{\vartheta}_j}_\text{nuovo peso}\leftarrow \vartheta_j + ( y^{(i)} -\hat{y}^{(i)})x_j
$$

Chiaramente, il percettrone non dovrà modificare i suoi pesi se $\hat{y}^{(i)} = y^{(i)}$. Questa formula lo contempla, in quanto $y^{(i)} - \hat{y}^{(i)} = 0$: il peso non verrà aggiornato. 
Si può aggiungere un parametro $\alpha$, detto \textbf{learning rate}. Il \underline{percettrone standard non lo prevede}.

$$
\hat{\vartheta}_j \leftarrow \vartheta_j + \alpha( y^{(i)} -\hat{y}^{(i)})x_j
$$

\noindent
Scelto in maniera opportuna, potrebbe migliorare l'apprendimento e la convergenza.

\subsection{Versione di Adaline}

Una versione successiva dell'algoritmo di ricalcolo dei pesi, per stabilire con maggiore precisione di quanto incrementare o decrementare i pesi, stabilisce che

$$
\hat{\vartheta}_j \leftarrow \vartheta_j + \left( y^{(i)} - \sum_{i}\vartheta_i \cdot x_i\right)x_j
$$

\noindent
Osserviamo che, per ogni esempio $i$-esimo, l'uscita desiderata $y^{(i)}$ può assumere solo due valori, $0$ oppure $1$. Allo stesso tempo, la somma pesata 

\[
\sum_i \vartheta_i \cdot x_i
\]

\noindent
rappresenta una combinazione lineare normalizzata degli ingressi, e pertanto assume valori compresi nell'intervallo $[0,1]$.  Ne consegue che la differenza tra il valore target $y^{(i)}$ e la somma pesata non potrà che appartenere all’intervallo $[-1,1]$. In altre parole, se il neurone commette l’errore massimo possibile, questo sarà pari a $1$ in valore assoluto, cioè la distanza più grande tra un’uscita binaria ($0$ o $1$) e un valore previsto all’interno di $[0,1]$.

\section{Modelli lineari}
Il percettrone funziona funziona correttamente solo se i dati sono \emph{linearmente separabili}, ovvero se esiste una frontiera lineare che separa le due classi. Ad esempio, il problema AND è linearmente separabile.

\subsection{Problema della separabilità non lineare}
\textit{Questa parte è un approfondimento teorico sulla separabilità lineare. Può essere saltata senza problemi.}

\paragraph{Definizione (separabilità lineare).}
Sia $X=\{x^{(1)},\ldots,x^{(m)}\}\subset\mathbb{R}^{d}$ e sia $y\in\{0,1\}^{m}$ un insieme di etichette.
Diciamo che i due insiemi di punti $A=\{x^{(i)}:y^{(i)}=1\}$ e $B=\{x^{(i)}:y^{(i)}=0\}$ sono \emph{linearmente separabili} se esistono $w\in\mathbb{R}^{d}$ e $b\in\mathbb{R}$ tali che
\[
w^\top x \;>\; b \quad \forall x\in A,
\qquad
w^\top x \;<\; b \quad \forall x\in B.
\]
Equivalente: esiste un iperpiano $w^\top x=b$ la cui parte positiva contiene tutti i punti di $A$ e la parte negativa tutti i punti di $B$.

\paragraph{Quante funzioni booleane sono linearmente separabili?}
Per $m$ argomenti binari esistono $2^{2^{m}}$ funzioni booleane possibili; solo una parte è realizzabile con un singolo classificatore lineare.
Dati noti:
\[
\begin{array}{lcl}
m=2 &\Rightarrow& 14 \text{ su } 16 \text{ sono separabili linearmente},\\[2pt]
m=3 &\Rightarrow& 104 \text{ su } 256 \text{ sono separabili linearmente},\\[2pt]
m=4 &\Rightarrow& 1882 \text{ su } 65536 \text{ sono separabili linearmente}.
\end{array}
\]
Non esiste una formula chiusa semplice che dia, in funzione di $m$, quante funzioni sono linearmente separabili; tuttavia si osserva che, all’aumentare di $m$, la \emph{frazione} di funzioni separabili decresce rapidamente.

\paragraph{Dicotomie di un insieme di punti.}
Dato $X=\{x^{(1)},\ldots,x^{(m)}\}\subset\mathbb{R}^{d}$, le possibili etichettature (o \emph{dicotomie}) sono $2^{m}$:
\[
\mathcal{Y}=\{0,1\}^{m} .
\]
Solo una parte di queste dicotomie è realizzabile mediante una frontiera lineare. Con $d$ fissata, il numero di dicotomie realizzabili cresce in modo polinomiale in $m$ (ordine al piu $m^{d}$), mentre il totale cresce esponenzialmente ($2^{m}$); di conseguenza,
\[
\Pr\{\text{dicotomia separabile linearmente}\}\;\longrightarrow\;0
\quad\text{quando } m \gg d .
\]

\begin{figure}[htbp]
	\centering
	\includegraphics[width=0.7\linewidth]{images/non_linear_separation-probability.png}
	\caption{Separabilità non lineare. A sinistra: andamento qualitativo della probabilità che una dicotomia sia linearmente separabile al crescere del numero di campioni \(m\) (con dimensione \(d\) fissata); a destra: esempio in \(\mathbb{R}^2\) di punti non separabili con un singolo iperpiano.}
	\label{fig:nonlinear-separability}
\end{figure}

Intuitivamente: se il numero di campioni cresce molto a parità di dimensione $d$, è sempre meno probabile che una separazione perfetta con una sola retta/iperpiano esista.

\subsection{Problema dello XOR}
Supponiamo però di voler computare lo XOR logico. Per farlo, costruiamo la tabella: 
\begin{center}
	\begin{tabular}{|c|c|c|}
		\hline
		\(x_1\) & \(x_2\) & \(XOR(x_1,x_2)\)\\
		\hline
		0 & 0 & 0\\
		0 & 1 & 1\\
		1 & 0 & 1\\
		1 & 1 & 0\\
		\hline
	\end{tabular}
\end{center}

\noindent
Il problema XOR non è linearmente separabile (immagine \ref{fig:xor-schema}): i quattro punti \((0,0),(1,0),(0,1),(1,1)\) con etichetta di uscita \(0,1,1,0\) non possono essere separati da una singola frontiera lineare. Un percettrone (che produce solo confini di decisione lineari) non riesce quindi a risolverlo.

\begin{figure}[htbp]
	\centering
	\includegraphics[width=.55\linewidth]{images/xor-schema.jpg}
	\caption{Schema del problema XOR: i vertici \((0,1)\) e \((1,0)\) appartengono alla classe \(1\), \((0,0)\) e \((1,1)\) alla classe \(0\). Nessuna frontiera lineare può separare le classi; è necessaria una frontiera non lineare (ottenibile con uno strato nascosto).}
	\label{fig:xor-schema}
\end{figure}

\paragraph{Dimostrazione della non-linearità dello XOR.} Assumendo la tabella di verità dello XOR e un percettrone con pesi \(v_1,v_2\) e soglia \(v_0\), le condizioni di attivazione diventano

\[
\begin{aligned}
	&1\cdot v_1 + 0\cdot v_2 > v_0,\\
	&1\cdot v_1 + 1\cdot v_2 < v_0,\\
	&0\cdot v_1 + 1\cdot v_2 > v_0,\\
	&0\cdot v_1 + 0\cdot v_2 < v_0.
\end{aligned}
\]

Da cui si ottiene
\[
2\,v_0 < v_1 + v_2 < v_0,
\]
ovvero una contraddizione: non esiste configurazione dei parametri di un singolo percettrone che risolva XOR.

\paragraph{Rete di percettroni.} La soluzione consiste nell’usare una \emph{rete di percettroni} (uno strato nascosto) che trasformi lo spazio in modo da rendere il problema linearmente separabile all’uscita. Un esempio minimale è:
\[
\begin{aligned}
	&f_1(x_1,x_2) = [\,x_1 - x_2 \ge 0.5\,],\\
	&f_2(x_1,x_2) = [\,x_2 - x_1 \ge 0.5\,],\\
	&f_3\big(f_1,f_2\big) = [\,1\cdot f_1 + 1\cdot f_2 \ge 0.5\,],
\end{aligned}
\]
dove \([\cdot]\) indica una funzione soglia (vale \(1\) se la condizione è vera, \(0\) altrimenti). Le funzioni \(f_1\) e \(f_2\) realizzano una trasformazione non lineare degli input; nello spazio così trasformato l’uscita \(f_3\) è ottenuta tramite una semplice soglia lineare.

\begin{figure}[tbph]
	\centering
	\includegraphics[width=0.7\linewidth]{images/neural-net-xor}
	\caption{Rete di percettroni per computare lo $XOR$}
\end{figure}


Infine, sebbene reti di percettroni possano risolvere $XOR$, l’algoritmo di apprendimento del singolo percettrone non è direttamente applicabile a reti multistrato; storicamente ciò contribuì al primo \emph{AI winter} e la situazione migliorò con la formalizzazione della \emph{retropropagazione} (\emph{backpropagation}). 

\chapter{Regressione}
Definiamo il compito di regressione in termini di come un algoritmo elabora un esempio
di input e produce un valore (o vettore) reale di output. Formalmente, vogliamo
apprendere una funzione
\[
h_\theta:\ \mathbb{R}^{d_1}\rightarrow \mathbb{R}^{d_2},\qquad d_1\ge 1,\ d_2\ge 1,
\]
a partire da un dataset supervisionato
\[
D=\{(x^{(i)},y^{(i)})\}_{i=1}^m,\qquad x^{(i)}\in\mathbb{R}^{d_1},\ y^{(i)}\in\mathbb{R}^{d_2}.
\]

\section{Ingredienti del task}
\begin{itemize}
  \item \textbf{Task}: predire valori reali a partire da input reali.
  \item \textbf{Modello}: ipotesi parametrica \(h_\theta\) che mappa input in output.
  \item \textbf{Dati}: coppie etichettate \((x^{(i)},y^{(i)})\).
  \item \textbf{Algoritmo di learning}: metodo per stimare \(\theta\) (ottimizzazione).
  \item \textbf{Funzione di loss}: misura lo scarto tra predizioni e target.
  \item \textbf{Valutazione}: metriche su validation/test per giudicare il modello.
\end{itemize}

\subsection{Tipi di regressione}
\begin{itemize}
  \item \textbf{Semplice (univariata)}: \(h_\theta:\mathbb{R}\to\mathbb{R}\) (\(d_1=d_2=1\)).
  \item \textbf{Multipla}: \(h_\theta:\mathbb{R}^{d_1}\to\mathbb{R}\) con \(d_1>1\).
  \item \textbf{Multivariata}: \(h_\theta:\mathbb{R}^{d_1}\to\mathbb{R}^{d_2}\) con \(d_2>1\) (più uscite).
\end{itemize}

\section{Regressione lineare semplice}
Nel caso \(d_1=d_2=1\) modelliamo la relazione tra un singolo ingresso \(x\in\mathbb{R}\) e
un'uscita reale \(y\in\mathbb{R}\) con una retta:
\[
h_\theta(x)=\theta_0+\theta_1\,x,
\]
dove \(\theta_0\) è l'intercetta e \(\theta_1\) la pendenza. ``Imparare'' significa scegliere
\(\theta=(\theta_0,\theta_1)\) in modo che le predizioni siano vicine ai corrispondenti target.

\subsection{Funzione di loss}
Misuriamo la qualità della retta con l'errore medio quadratico (MSE):
\[
J(\theta)=\frac{1}{2m}\sum_{i=1}^{m}\big(h_\theta(x^{(i)})-y^{(i)}\big)^2.
\]
Il fattore \(1/2\) è una costante di comodo che semplifica le derivate; il quadrato rende
positivi tutti i contributi ed enfatizza gli errori grandi. In regressione lineare \(J\) è
convessa rispetto ai parametri, quindi il minimo è unico (se c'è variabilità negli \(x^{(i)}\)).

\section{Regressione lineare multipla}
Estendiamo al caso con più variabili in ingresso. Dato il vettore
\(x=(x_1,\dots,x_n)\), usiamo un parametro per ogni dimensione più il bias:
\[
f(x)=\theta_0+\theta_1x_1+\theta_2x_2+\cdots+\theta_n x_n .
\]
Per semplicità poniamo \(x_0=1\) e definiamo il vettore esteso
\(
x=(x_0,x_1,\dots,x_n)^\top
\);
allora
\[
f(x)=\sum_{i=0}^n \theta_i x_i \;=\; \boldsymbol{\theta}^\top x.
\]
Questo modello è, in sostanza, un \emph{percettrone senza soglia}: la combinazione lineare
\(\boldsymbol{\theta}^\top x\) è l'uscita continua del regressore.

\section{Feature scaling}
Per far funzionare bene (e in fretta) la discesa del gradiente, le feature vanno portate
su scale simili. Due opzioni comuni:
\[
\text{z-scoring:}\quad x_j\leftarrow\frac{x_j-\mu_j}{\sigma_j},
\qquad
\text{min--max:}\quad x_j\leftarrow\frac{x_j-x_j^{\min}}{x_j^{\max}-x_j^{\min}}.
\]
Le statistiche \((\mu_j,\sigma_j,x_j^{\min},x_j^{\max})\) si calcolano solo sul training e
si riusano (senza ricalcolarle) su validation/test, per evitare data leakage.

\section{Algoritmo di discesa del gradiente}
L’obiettivo è scegliere \(\theta\) che minimizza \(J(\theta)\). La \emph{discesa del gradiente}
è un metodo iterativo che aggiorna i parametri nella direzione di massima diminuzione di \(J\).

\subsection{Versione batch per la regressione lineare}
Sia \(X\in\mathbb{R}^{m\times(n+1)}\) la design matrix con prima colonna di \(1\),
\(y\in\mathbb{R}^{m}\) il vettore dei target e \(\hat y=X\theta\) le predizioni. La loss è
\[
J(\theta)=\frac{1}{2m}\|X\theta-y\|_2^2,\qquad
\nabla_\theta J(\theta)=\frac{1}{m}X^\top(X\theta-y).
\]
\paragraph{Pseudocodice.}
\begin{enumerate}
  \item \textbf{Inizializza} in modo casuale: \(\theta^{(0)}=(\theta_0,\ldots,\theta_n)^\top\).
  \item \textbf{Calcola le derivate parziali}:
  \[
  \frac{\partial J(\theta)}{\partial \theta_j}
  =\frac{1}{m}\sum_{i=1}^m\big(\theta^\top \tilde x^{(i)}-y^{(i)}\big)\,\tilde x^{(i)}_j,\qquad j=0,\ldots,n,
  \]
  dove \(\tilde x^{(i)}=(1,x^{(i)}_1,\ldots,x^{(i)}_n)^\top\).
  \item \textbf{Aggiorna} i parametri (passo di ampiezza \(\alpha>0\)):
  \[
  \theta_j \leftarrow \theta_j - \alpha\,\frac{\partial J(\theta)}{\partial \theta_j},
  \qquad j=0,\ldots,n.
  \]
  \item \textbf{Ripeti} i passi 2–3 finché non è soddisfatto un criterio d’arresto
  (iterazioni massime, \(\|\nabla J(\theta)\|\) sotto soglia, variazione di \(J\) piccola).
\end{enumerate}
In forma compatta: \(\;\theta \leftarrow \theta - \frac{\alpha}{m}X^\top(X\theta-y)\).

\paragraph{Equazioni normali (soluzione chiusa).}
Se \(X^\top X\) è invertibile, il minimo di \(J\) è
\[
\theta^\star=(X^\top X)^{-1}X^\top y.
\]
Utile per problemi piccoli; per \(n\) o \(m\) grandi conviene la discesa del gradiente.

\section{Regressione polinomiale e feature mapping}
Se la relazione input–output è non lineare, usiamo un mapping
\(\phi:\mathbb{R}^{n}\to\mathbb{R}^{p}\) (poteri e interazioni) e poi una regressione
\emph{lineare nei parametri} nello spazio trasformato:
\[
h_\theta(x)=\theta^\top \phi(x).
\]
Esempio (grado \(2\), due feature):
\[
\phi(x_1,x_2)=\big(1,\ x_1,\ x_2,\ x_1^2,\ x_2^2,\ x_1x_2\big).
\]
Gradi più alti aumentano la capacità (meno underfitting) ma anche il rischio di overfitting
e i costi: il numero di termini cresce rapidamente con \(n\) e con il grado.

\section{Regolarizzazione}
Per ridurre l’overfitting penalizziamo pesi grandi. In \textbf{Ridge} (L2) la loss è
\[
J_\lambda(\theta)=\frac{1}{2m}\|X\theta-y\|_2^2+\frac{\lambda}{2m}\sum_{j=1}^{n}\theta_j^2,
\]
dove tipicamente \(\theta_0\) non si penalizza. Aggiornamenti:
\[
\theta_0\leftarrow \theta_0-\alpha\,\frac{1}{m}\sum_{i=1}^m(\hat y^{(i)}-y^{(i)}),\qquad
\theta_j\leftarrow \theta_j-\alpha\left[\frac{1}{m}\sum_{i=1}^m(\hat y^{(i)}-y^{(i)})\tilde x^{(i)}_j+\frac{\lambda}{m}\theta_j\right]\ (j\ge 1).
\]

\section{Regressione multivariata}
Se \(d_2>1\), stimiamo più uscite in parallelo. Con \(Y\in\mathbb{R}^{m\times d_2}\) (righe \(y^{(i)\top}\))
e \(\Theta\in\mathbb{R}^{(n+1)\times d_2}\), il modello è \(\hat{Y}=X\Theta\). L’MSE matriciale è
\(J(\Theta)=\tfrac{1}{2m}\|X\Theta-Y\|_F^2\). Le colonne di \(\Theta\) (una per uscita) si
ottimizzano in modo indipendente; la soluzione chiusa generalizza a
\[
\Theta^\star=(X^\top X+\lambda \tilde M)^{-1}X^\top Y\qquad\text{(con Ridge opzionale)}.
\]

\section{Valutazione}
Metriche tipiche:
\[
\text{MSE}=\frac{1}{m}\sum_{i=1}^m\!\big(\hat y^{(i)}-y^{(i)}\big)^2,\quad
\text{RMSE}=\sqrt{\text{MSE}},\quad
\text{MAE}=\frac{1}{m}\sum_{i=1}^m\!\lvert \hat y^{(i)}-y^{(i)}\rvert.
\]
\noindent
\textbf{\(R^2\)} (coefficiente di determinazione):
\[
R^2=1-\frac{\sum_i\big(y^{(i)}-\hat{y}^{(i)}\big)^2}{\sum_i\big(y^{(i)}-\bar{y}\big)^2}.
\]
\textbf{REC curve}: ordinando gli errori assoluti \(e_i=\lvert \hat y^{(i)}-y^{(i)}\rvert\) e
tracciando, al variare di \(\varepsilon\), la frazione cumulativa di esempi con errore
\(\le\varepsilon\), si ottiene una curva di facilissima lettura; un’AUC maggiore indica in
genere prestazioni migliori.
\chapter{Classificazione}

\section{Introduzione alla classificazione}
La \textbf{classificazione} è un altro dei problemi tipici che può essere risolto tramite un algoritmo di Machine Learning. Riguarda lo \textbf{stabilire se un dato input appartiene ad una tra delle classi} prestabilite dal problema. Il\textbf{ target non sarà più un valore} $\in \mathbb{R}$, ma una tra $k$ classi.
Per studiare la classificazione, andremo a stabilire i requisiti e i nostri obiettivi.
\begin{itemize}
	\item Un modello ben definito.
	\item Una funzione di loss per stabilire lo scarto con i risultati attesi.
	\item Un algoritmo di learning per trovare i parametri.
	\item Delle misure di valutazione.
	\item Niente overfitting!
\end{itemize}

\subsection{Classificazione: binaria e di classe}
Identifichiamo due tipi di task di classificazione:

\begin{itemize}
	\item \textbf{Classificazione binaria.}\\
	Le etichette $y \in \{ 0,1\}$. Va associata ad $x$ una delle due etichette, o le probabilità relativa a ciascuna delle due.
	\item \textbf{Classificazione di classe.}\\
	Le etichette $y \in \{ 0,1, \dots, k-1\}$. Va associata ad $x$ una delle etichette, o la probabilità relativa a ciascuna di esse.
\end{itemize}

È fondamentale sottolineare l'importanza che gli output di questi algoritmi siano sempre $\in[0,1]$, in quanto da interpretare come \textbf{probabilità}.
\section{Classificazione binaria}
Nonostante i classificatori binari possano sembrare limitati, questi trovano applicazioni in vari task: solitamente, questi si basano sulla suddivisione di immagini più grandi in \textit{patches} più piccole, con una successiva, appunto, classificazione, per individuare determinati elementi. Nel \textbf{medical imaging}, i classificatori binari possono individuare piccoli tumori. Nei \textbf{controlli qualità }in ambito industriale, possono individuare \textbf{imperfezioni} nei materiali. Nel campo più generale della \textbf{computer vision}, gli algoritmi di classificazione (binaria e non) sono fondamentali.

\subsection{Perché non usare la regressione lineare per la classificazione?}

Si potrebbe pensare di \textbf{usare un modello di regressione lineare per effettuare classificazione binaria}, e l'intuizione non sarebbe nemmeno totalmente sbagliata: potremmo usare, ad esempio, la regressione per \textbf{trovare una retta che unisce i dati }in maniera opportuna, e in funzione della pendenza di questa retta, \textbf{stabilire un valore di sogliatura }per distinguere due classi.

Forniamo il seguente esempio: vogliamo addestrare un modello per prevedere se un tumore è benigno o maligno usando una singola feature (dimensione del tumore).

Usiamo la regressione lineare e facciamo il fitting di una retta nello spazio (troviamo la retta, cioè la funzione che meglio si posiziona a media tra tutti i punti). Possiamo trovare il miglior $y = \vartheta^Tx$ dal nostro set di dati e quindi selezionare una \textbf{soglia} su $y$ per classificare quando il tumore è maligno o meno:
$h_\vartheta(x) \leq 0.5 \to 0, \quad h_\vartheta(x) \geq 0.5 \to 1$.

\begin{figure}[tbph]
	\centering
	\includegraphics[width=0.7\linewidth]{images/regression-for-classification.png}
\end{figure}



Ci sono \textbf{due criticità} relative all'utilizzo della regressione. La prima, riguarda la versatilità del modello: il fitting della retta non è opportuno, in  quanto la regressione lineare dipende fortemente dalla distribuzione dei dati.


\begin{figure}[tbph]
	\centering
	\includegraphics[width=0.7\linewidth]{images/regression-for-classification2}
\end{figure}
\newpage
La seconda criticità riguarda invece la $y$, non sempre limitata tra 0 e 1, come ci aspettiamo in un modello di classificazione binaria. Come ci comportiamo con valori più grandi di 1? E con valori minori di 0? 

\section{Regressione logicistica - Sigmoide}
Se la regressione lineare restituisce una $y$ non limitata tra 0 e 1, la \textbf{regressione logistica} risolve il problema con una funzione, detta \textbf{sigmoide}, applicata sul risultato $z = \vartheta^TX$ del modello di regressione. La \textbf{sigmoide} una funzione del tipo

$$
\overline{\sigma}(z) = \frac{1}{1 + e^{-z}} = \frac{1}{1 + e^{-\vartheta^TX}}
$$
\noindent
con $z = \vartheta_0 + \vartheta_1 x_1 + \dots + \vartheta_i x_i$. Questa funzione ha un'interpretazione probabilistica molto banale, opportuna per \textbf{adattare i modelli di regressione lineare su problemi di classificazione}\footnote{Non si può pretendere di usare un modello di regressione lineare su task di classificazione, ma è comunque possibile, tramite l'uso della sigmoide, sfruttarlo per task di classificazione binaria.}. Osserviamo che:

\begin{itemize}
	\item $1-\overline{\sigma} = \sigma(z)$
	\item $\displaystyle\frac{\partial \overline{\sigma}(z)}{\partial t} =\overline{\sigma}(z)(1-\overline{\sigma}(z)) = \overline{\sigma}(z)\overline{\sigma}(-z)$
	\item È inoltre dimostrabile che $\displaystyle\overline{\sigma} = \frac{1}{2} + \frac{1}{2}\tanh\left(\frac{z}{2}\right)$. Questo ci apre le porte a implementazioni molto più semplici.
\end{itemize}

Concludiamo che la regressione logicistica, è ottimale per trovare opportuni parametri di $\vartheta^T$, che moltiplicati a $x$, ci permetteranno di stabilire il \textbf{decision boundary}, nella ben nota forma:

$$
\vartheta_0x_0 + \vartheta_1x_1 + \vartheta_2x_2 + \dots \vartheta_dx_d  
$$

\begin{figure}[tbph]
	\centering
	\includegraphics[width=0.75\linewidth]{images/sigmoid}
\end{figure}


\subsection{Vantaggi relativi all'uso della sigmoide}

\begin{itemize}
	\item Interpretazione probabilistica semplice. Da valori $\in[0,1]$.
	\item Derivabile, più liscia rispetto ad una step function.
	\item Introduce non lineareità, migliorando la classificazione.
\end{itemize}

\section{Funzione di Loss}
Avere una funzione di Loss associata al modello è fondamentale per capire come effettuare il training, e quali sono i migliori parametri di $\vartheta^T$ per la classificazione. 
Per modellare questa funzione (da minimizzare secondo l'algoritmo di discesa del gradiente), utilizzeremo due funzioni logaritmiche. La funzione di Loss sarà definita in funzione dell'etichetta $\hat{y}$.
$$
\text{Loss}(h_\vartheta(x), y) = \begin{cases}
	- \log (h_\vartheta(x)) \qquad\qquad\text{se } \hat{y} = 1\\
	- \log (1-h_\vartheta(x)) \qquad\text{ se } \hat{y} = 0\\
\end{cases}
$$

Questo, perché l'errore deve essere 1 quando $\hat{y} = 0$ e $y = 1$, o quando $\hat{y} = 1$ e $y = 0$.

\begin{center}
	\textit{Ai fini della semplicità e univocità della spiegazione, useremo $y$ per indicare ciò che fino ad ora abbiamo indicato con $\hat{y}$.}
\end{center}


\begin{figure}[tbph]
	\centering
	\includegraphics[width=0.50\linewidth]{images/loss-logistic-regression}
\end{figure}
\noindent
Possiamo esprimere la funzione di Loss anche in questo modo, rendendo unica la definizione della funzione.
$$
\text{Loss}(h_\vartheta(x), y) = -[y\log (h_\vartheta(x)) + (1-y)\log (1 - h_\vartheta(x))]
$$
\noindent
Definita la funzione di Loss, andremo a definire la funzione di costo

$$
J(\vartheta) = - \frac{1}{m} \sum_{i=1}^{m}(y^{(i)}\log (h_\vartheta(x)) + (1-y^{(i)})\log (1 - h_\vartheta(x)))
$$

\noindent
su cui sarà possibile applicare l'algoritmo di discesa del gradiente, fino alla convergenza del modello.
$$
\vartheta_j^{\text{new}} = \vartheta_j^{\text{old}} -\alpha\frac{\partial J(\vartheta)}{\partial \vartheta_j} \qquad \forall j
$$

\subsection{Derivata parziale della funzione di costo}

Necessaria per applicare la discesa del gradiente.
$$
\frac{\partial J(\vartheta)}{\partial \vartheta_j} = \frac{1}{m} \sum_{i=1}^{m} (\underbrace{h_\vartheta(x^{(i)})}_{(*)}- y^{(i)}))x_j^{(i)}) \qquad \forall j = 0, \dots, d
$$
(*) Questo membro nasconde la funzione sigmoide $\overline{\sigma}(\vartheta^TX^{(i)})$.

\subsection{Entropia dell'informazione}

La loss function di questo modello presenta un'interpretazione probabilistica basata sul concetto di \textbf{entropia} dell'informazione. Parleremo quindi \textbf{Binary Cross Entropy Loss}. La seguente, è la formula dell'entropia:

$$
H(X) = -\sum_{k=1}^{K} p_k\log_2(p_k)
$$

con $p_k$ probabilità di essere di classe $k$.  La distribuzione di probabilità (e quindi l'entropia). Queste distribuzioni informazioni su:
\begin{itemize}
	\item Incertezza.
	\item Misura del disordine dalle classi.
	\item Quanto la probabilità è concentrata su una classe.
	\item Informazione (la distribuzione è molto informativa se è capace di distinguere in maniera opportuna).
\end{itemize}
Un esempio su $k=2$ è la funzione 
$$
H(X) = - p_1\log_2(p_1) - p_2\log_2(p_2) = -[p_1\log_2(p_1) + p_2\log_2(p_2) ]
$$
\begin{figure}[tbph]
	\centering
	\includegraphics[width=1\linewidth]{images/cross_entropy}
\end{figure}

\textit{$\log_2(0)$ è indefinito. Quando facciamo calcoli relativi alla cross-entropy, consideriamo valori piccoli, mai nulli. In questo esempio e nel prossimo esercizio, quando utilizziamo 0, stiamo approssimando un valore molto piccolo, ma mai effettivamente nullo. Inoltre, in un modello reale, avere una $p_k = 0$ o $p_k = 1$ significa probabile overfitting del modello.}
\newpage

\subsubsection{Esercizio sull'entropia}
Riordina le seguenti distribuzioni in ordine crescente di entropia.

\begin{figure}[tbph]
	\centering
	\includegraphics[width=1\linewidth]{entropia-esercizio}
\end{figure}

$$
H(X) = - p_1\log_2(p_1) - p_2\log_2(p_2) = -[p_1\log_2(p_1) + p_2\log_2(p_2) + p_3\log_2(p_3) ]
$$

\begin{enumerate}
	\item $H(X_1) =  -[0\log(0) + 1\log(1) + 0\log(0) ] = 0$
	\item $H(X_2) =  -[1\log(1) + 0\log(0) + 0\log(0) ] = 0$
	\item $H(X_3) =  -[0,33\log(0,33) + 0,33\log(0,33) + 0,33\log(0,33) ] \approx 1,5848$
	\item $H(X_4) =  -[0,80\log(0,80) + 0,20\log(0,20) + 0\log(0) ] \approx 0,722$
\end{enumerate}

$$
X_1 = X_2 < X_4 < X_3
$$

\subsection{Binary Cross Entropy Loss}

La seguente formula è equivalente alla nostra funzione di Loss, ma fornisce un'interpretazione più intuitiva. Definiamo la \textbf{Binary Cross Entropy Loss} come:

$$
H_{BCE}(X) = \frac{1}{m}\sum_{i=1}^{m} \left(-\sum_{k=1}^{K}\underbrace{p_k^{(i)}}_{\text{(a)}} \log_2(\underbrace{q_k^{(i)}}_{\text{(b)}})\right)
$$

Dove (a) è la distribuzione vera della classificazione, detta \textbf{ground truth}, ovvero quella con probabilità massima sulla classe attesa, mentre (b) è la distribuzione data dal classificatore. Misura quindi la \textbf{distanza tra le due distribuzioni}. Minimizzare $H_{BCE}(X)$, significherà rendere le distribuzioni ideali e quelle effettive del classificatore, quanto più simili possibile.

\begin{figure}[tbph]
	\centering
	\includegraphics[width=0.9\linewidth]{images/coss-entropy2}
\end{figure}


\newpage

\section{Overfitting nella classificazione}
Rischiamo di cadere in overfitting nell'utilizzo di classificatori polinomiali. Basterà usare i termini di regolarizzazione per diminuire o annullare l'impatto di alcuni termini di grado eccessivamente alto, ottenendo così:
$$
J(\vartheta) = - \frac{1}{m} \sum_{i=1}^{m}(y^{(i)}\log (h_\vartheta(x)) + (1-y^{(i)})\log (1 - h_\vartheta(x))) + \frac{\lambda}{2m}\sum_{j=1}^n\vartheta_j^2
$$ 

\begin{figure}[tbph]
	\centering
	\includegraphics[width=1\linewidth]{images/underoverfitting-classification}
\end{figure}

\section{Reject Region, regione d'incertezza}

Sia nella classificazione binaria, che in quella multiclasse, è possibile osservare casi in cui la probabilità stimata che un input appartenga ad una tra le classi specificate, non superi un determinato valore di certezza. Sono dei casi limite che vanno gestiti in maniera opportuna:
\begin{enumerate}
	\item Prendere il valore con certezza più alta (se è proprio obbligatorio stabilire una classe, e il task non è particolarmente delicato).
	\item Scartare l'input.
\end{enumerate}

\begin{figure}[tbph]
	\centering
	\includegraphics[width=0.47\linewidth]{images/rejection-region}
	\caption{La zona d'incertezza (in cui $P(C_1|x) \sim P(C_2|x)$)}
\end{figure}


\newpage

\section{Classificazione multiclasse e one vs all}  
La classificazione multiclasse si distingue da quella binaria per il numero di classi.
Il dataset di input non avrà più etichette binarie, ma $k$ etichette. A ogni classe si associa un numero, ai fini di semplicità. 

\subsection{Metodo One vs All}
Possiamo ottenere una classificazione multiclasse da dei classificatori binari, usando la metodologia \textbf{one vs all}. Immaginiamo di avere un sistema di classificazione figure geometriche, con le classi \textit{triangolo}, \textit{quadrato }e \textit{cerchio }($c_1,c_2,c_3$).
Avremo bisogno di tre classificatori, $h_{\vartheta}^1, h_{\vartheta}^2, h_{\vartheta}^3$, capaci di misurare $P(\text{triangolo}|x), P(\text{quadrato}|x), $ $P(\text{cerchio}|x)$. Prenderemo poi l'etichetta associata al valore di probabilità più alto

$$
\hat{k} = \text{arg}\max_k h_{\vartheta}^k(\overline{x})
$$

ottenendo effettivamente una classificazione multiclasse. 

\begin{figure}[tbph]
	\centering
	\includegraphics[width=0.8\linewidth]{images/onevsrest}
\end{figure}

Il training è effettuato sui singoli classificatori binari.
Osserviamo inoltre che, con $k$ classi:

$$
\text{Numero di classificatori richiesti in one vs all} = k 
$$

\subsection{Metodo One vs One}
In questo caso, i classificatori distinguono classi a due a due. Si fanno poi le opportune valutazioni per capire quale classe assegnare. Il numero di classificatori risulta essere più alto. Con $k$ classi abbiamo:

$$
\text{Numero di classificatori richiesti in one vs one} = \frac{k(k-1)}{2}
$$

Immaginiamo un classificatore per lo stesso problema precedentemente esposto: otteniuamo le classi $(c_1,c_2,c_3)$. Questi classificatori su coppie, daranno poi degli opportuni risultati:
$$
c_1 \text{ vs } c_2, \quad c_2 \text{ vs } c_3, \quad c_1 \text{ vs } c_3
$$

\begin{figure}[tbph]
	\centering
	\includegraphics[width=0.75\linewidth]{images/onevsone}
\end{figure}

La classe con più voti è quella predetta. 

\subsection{Confronto tra \textit{One vs All} e \textit{One vs One}}
Il modello one vs one, per quanto più oneroso in termini di numero di classificatori binari richiesti, non presenta l'area di incertezza su tutte le classi, che invece possiamo trovare nel classificatore one vs all.

\begin{figure}[tbph]
	\centering
	\includegraphics[width=1\linewidth]{images/onevsonevsonevsall}
\end{figure}


\chapter{Design, valutazione e scelta dei parametri di un algoritmo di ML}

\section{Momento e gradiente}
Una variante dell'algoritmo di discesa del gradiente, ovvero \textbf{la discesa del gradiente con momento}, modifica la nostra formula del gradiente nel seguente modo:

$$
\vartheta_j^\text{new} \leftarrow \vartheta_j^\text{old} - \alpha z^\text{new}
$$

$$
z^\text{new} \leftarrow \beta z^\text{old} + \frac{\partial J(\vartheta^\text{old})}{\partial \vartheta^\text{old}} 
$$

l'idea è quella di mantenere in $z$ una \textbf{media dei gradienti recenti}, in modo da mantenere coerenza durante la convergenza: valori incoerenti del gradiente sono \textbf{rumore}, e questo \textbf{può causare zig-zag e rallentamenti nella convergenza}. Tenendo uno storico dei valori del gradiente (usando il \textbf{momentum} $\beta$ come parametro che stabilisce l'influenza dei vecchi valori su quelli nuovi), diventa possibile \textbf{annullare parzialmente o totalmente l'effetto del rumore}. Accelera anche la convergenza stessa, \textbf{rinforzando i passi coerenti}. Si accorcia quindi il training, e si migliorano le performance a parità di epoche. Osservazioni statistiche consigliano l'utilizzo di $\beta = 0.90$ per ottenere risultati migliori.


\section{Design di Algoritmi}
Progettare bene un algoritmo di ML permette di ottenere migliori risultati in meno tempo e con meno risorse computazionali. Si pensi al cercare di inserire a casaccio i parametri di un modello, senza una logica precisa: si rischia di perdere tempo, se invece si pensa a come variare i parametri in maniera sistematica, per capire l'impatto di ciascuno di essi sul modello si potrebbe ottenere un modello migliore in meno tempo.

\subsection{Strategie di miglioramento}
Supponiamo di aver implementato la regressione lineare con regolarizzazione, ma gli errori su nuovi campioni sono comunque troppo alti. Potremmo pensare ad alcune strategie per migliorare il modello:
\begin{description}
	\item[Aumentare i campioni di Training.] Non è sempre una scelta utile, anche perché se il modello non funziona su nuovi campioni significa che il problema (molto probabilmente) è di overfitting. Aumentare i campioni di training potrebbe aiutare, ma non è detto che lo faccia.
	\item[Ridurre il set di Features.] Si potrebbe pensare di ridurre il numero di features in input al modello, in modo da semplificare il problema e ridurre il rischio di overfitting. Questa tecnica può essere efficace se alcune features sono ridondanti o non rilevanti per il task. Tuttavia, è importante fare attenzione a non eliminare informazioni utili. 
	\item[Cercare nuove features.] L'opposto della riduzione delle features: il problema è di underfitting, in quanto il modello non ha abbastanza informazioni per generalizzare bene. Aggiungere nuove features può aiutare a migliorare le prestazioni del modello ma richiede un'attenta selezione delle stesse per evitare di introdurre rumore.
	\item[Utilizzare features polinomiali.] Trasformare le features esistenti in polinomiali può aiutare a catturare relazioni non lineari tra le variabili. Questa tecnica può essere utile se si sospetta che il modello lineare non sia sufficiente per rappresentare i dati. Tuttavia, l'aggiunta di termini polinomiali aumenta la complessità del modello e il rischio di overfitting, quindi è importante bilanciare questa scelta con tecniche di regolarizzazione.
	\item[Far variare il parametro di regolarizzazione $\lambda$.] Modificare il parametro di regolarizzazione può influenzare significativamente le prestazioni del modello. Un valore più alto di $\lambda$ penalizza maggiormente i pesi del modello, riducendo il rischio di overfitting, mentre un valore più basso consente al modello di adattarsi meglio ai dati di training, ma aumenta il rischio di overfitting. È importante trovare un equilibrio ottimale attraverso tecniche come la validazione incrociata (che vedremo più avanti). 
\end{description}

Solitamente si cerca di effettuare test di diagnostica applicati al modello, per permettere di ottenere informazioni dettagliate su cosa funziona o non funziona con l'obiettivo di ottenere indicazioni per migliorare le prestazioni.

\section{Valutazione}
La valutazione di un algoritmo di ML è fondamentale per capire se il modello è adatto al task. 

\subsection{Valutazione incrociata}
Un modo di fare valutazione incrociata\footnote{Per valutazione incrociata si intende la tecnica di suddividere il dataset in più parti per allenare e testare il modello in modo iterativo, garantendo una stima più robusta delle prestazioni.} è la \textbf{grid search}, che sfrutta la k-Fold Cross Validation: si suddivide il dataset in $t$ splits e in $k$ folds, e si allena il modello su $t-1$ splits e si valuta su uno split di validation, ripetendo il processo per tutti i $t$ splits. In questo modo si ottengono $t$ stime delle prestazioni del modello, che vengono poi mediate per ottenere una stima complessiva. Questo processo viene ripetuto per tutti i possibili set di iperparametri del modello, selezionando quelli che producono la \emph{migliore prestazione} media sul validation set. Infine, si valuta il modello finale con gli iperparametri selezionati sul test set per ottenere una stima finale delle prestazioni del modello.

\paragraph{Procedura.}
La procedura è la seguente:
\begin{enumerate}
	\item Si suddivide il dataset originale in training/validation set e un test set che non verrà usato fino alla fine della valutazione.
	\item Si suddivide il dataset rimanente in $t$ splits e in $k$ folds.
	\item Si seleziona uno split come validation set e gli altri $t-1$ come training set.
	\item Si allena il modello sui $t-1$ training splits.
	\item Si valuta il modello con una funzione $J_{\text{VAL}}$ sul validation fold (non usato per il training).
	\item Si ripete il processo per tutti i $t$ splits, ottenendo $t$ valori di $J_{\text{VAL}}$.
	\item Si calcola la media dei $t$ valori di $J_{\text{VAL}}$ per ottenere una stima complessiva delle prestazioni del modello.
	\item Si ripete l'intero processo per tutti i possibili set di iperparametri del modello.
	\item Si selezionano gli iperparametri che hanno prodotto la migliore prestazione media sul validation set.
	\item Infine, si valuta il modello finale con gli iperparametri selezionati sul test set per ottenere una stima finale delle prestazioni del modello.
\end{enumerate}

In questo modo il modello è robusto perché è stato allenato su diversi training set per i dati, allenato sui validation set per gli iperparametri e testato su un test set \textbf{mai visto prima}.

\paragraph{Esempio.}
Ipotizziamo di avere un certo dataset, dividiamolo in un test set (20\% dei dati) e in un training/validation set (80\% dei dati). Definiamo le funzioni di costo:
\begin{itemize}
	\item \textbf{Funzione di costo di training:}
	\[
	J_{\text{TR}}(\vartheta) = \frac{1}{2m_{\text{TR}}} \sum_{i=1}^{m_{\text{TR}}} \left( h_\vartheta(x^{(i)}_{\text{TR}}) - y^{(i)}_{\text{TR}} \right)^2
	\]
	Dove $m_{\text{TR}}$ è il numero di campioni nel training set, $x^{(i)}_{\text{TR}}$ e $y^{(i)}_{\text{TR}}$ sono rispettivamente l'input e l'output del $i$-esimo campione del training set.
	\item \textbf{Funzione di costo di validation:}
	\[
	J_{\text{VAL}}(\vartheta) = \frac{1}{2m_{\text{VAL}}} \sum_{i=1}^{m_{\text{VAL}}} \left( h_\vartheta(x^{(i)}_{\text{VAL}}) - y^{(i)}_{\text{VAL}} \right)^2
	\]
	Dove $m_{\text{VAL}}$ è il numero di campioni nel validation set, $x^{(i)}_{\text{VAL}}$ e $y^{(i)}_{\text{VAL}}$ sono rispettivamente l'input e l'output del $i$-esimo campione del validation set.
	\item \textbf{Funzione di costo di test:}
	\[
	J_{\text{TEST}}(\vartheta) = \frac{1}{2m_{\text{TEST}}} \sum_{i=1}^{m_{\text{TEST}}} \left( h_\vartheta(x^{(i)}_{\text{TEST}}) - y^{(i)}_{\text{TEST}} \right)^2
	\]
	Dove $m_{\text{TEST}}$ è il numero di campioni nel test set, $x^{(i)}_{\text{TEST}}$ e $y^{(i)}_{\text{TEST}}$ sono rispettivamente l'input e l'output del $i$-esimo campione del test set.
\end{itemize}

\noindent
Supponiamo di voler utilizzare $t=5$ splits e $k=5$ folds. La suddivisione del training/validation set sarà la seguente:

\begin{figure}[htbp]
	\centering
	\includegraphics[width=0.9\textwidth]{images/splits_folds_example.png}
	\caption{Dataset diviso in split e folds per la valutazione incrociata.}
\end{figure}

Dopo aver suddiviso i dati, ognuno dei dati produrrà una stima di $J_{\text{VAL}}^H(\vartheta^{(i)})$ per un certo $i$-esimo set valutata sull'iperparametro $H$. Da queste andiamo a fare la media:
\[
\bar{J}_{\text{VAL}}^H(\vartheta) = \frac{1}{t} \sum_{i=1}^{t} J_{\text{VAL}}^H(\vartheta^{(i)})
\]

Ottenuta la media $\bar{J}_{\text{VAL}}^H(\vartheta)$ per ogni set di iperparametri $H$, si seleziona quello che minimizza la funzione di costo media:
\[
H^* = \arg\min_H \bar{J}_{\text{VAL}}^H(\vartheta)
\]

Una volta fissato il miglior iperparametro $H^*$, si allena il modello sul test set per ottenere la stima finale:
\[
J_{\text{TEST}}^{H^*}(\vartheta)
\]

\subsection{Bias e Varianza}
Per valutare le prestazioni di un modello possiamo usare i valori del bias e della varianza per fare delle stime su come il modello si comporta sui dati.

\begin{figure}[htbp]
	\centering
	\includegraphics[width=\textwidth]{images/ovfit_underfit_rightfit.png}
	\caption{Underfitting---capacità adeguata---overfitting in regressione polinomiale: punti blu \(=\) dati di training (\(\mathrm{TR}\)), punti rossi \(=\) fuori da \(\mathrm{TR}\); la linea verde mostra il fit del modello: lineare (sinistra), polinomio di grado \(2\) (centro) e polinomio di grado \(5\) (destra).}
	\label{fig:ovfit_underfit_rightfit}
\end{figure}

\paragraph{Bias.} Ricordiamo che il bias è \textbf{l'errore sistematico} che il modello commette sui dati. Un modello con alto bias tende a sottostimare la complessità del problema, portando a errori elevati sia sul training set che sul test set. Questo fenomeno è noto come \textbf{underfitting}. Nell'immagine \ref{fig:ovfit_underfit_rightfit}, il grafico a sinistra mostra un esempio di underfitting, dove il modello lineare non riesce a catturare la relazione tra le variabili e commette un errore sistematico sia sul training set (punti blu) che sul test set (punti rossi).

\paragraph{Varianza.} La varianza rappresenta la sensibilità del modello alle variazioni nei dati di training. Un modello con alta varianza tende a sovradattarsi ai dati di training, catturando il rumore invece della vera relazione tra le variabili. Questo porta a errori bassi sul training set ma elevati sul test set, fenomeno noto come \textbf{overfitting}. 

\begin{figure}[htbp]
	\centering
	\includegraphics[width=0.75\textwidth]{images/bias_variance_tradeoff.png}
	\caption{Trade-off tra bias e varianza in funzione della complessità del modello.}
	\label{fig:bias_variance_tradeoff}
\end{figure}
 
Nella figura \ref{fig:ovfit_underfit_rightfit}, il grafico a destra mostra un esempio di overfitting, dove il modello polinomiale è troppo complesso e si adatta troppo strettamente ai dati di training, risultando in prestazioni scadenti sui dati di test\footnote{Si noti che basterebbe rimuovere un punto del training set per far cambiare completamente il modello, in quanto altamente sensibile ai dati di input.}.
Come si vede dall'immagine al centro della figura \ref{fig:ovfit_underfit_rightfit}, un modello con capacità adeguata riesce a bilanciare bias e varianza, ottenendo buone prestazioni sia sul training set che sul test set. Per valutare e risolvere i problemi sul modello, possiamo andare a controllare come Bias e Varianza variano al variare dei parametri del modello\footnote{In quella che molti chiamano \textbf{bias-variance trade-off}.} e notiamo che si può costruire una correlazione tra l'errore commesso dal modello (quindi o in $J_{\text{TEST}}$ o in $J_{\text{VAL}}$) e la complessità del polinomio (il grado):


\paragraph{Parametro di regolarizzazione.}

Anche i parametri di regolarizzazione, esattamente come il modello (i suoi parametri), devono essere stabiliti sperimentalmente. Si parte da $\lambda^0 = 0$, testando valori di $\lambda^{(1)}, \lambda^{(2)} \ldots \lambda^{(k)}$ sempre più grandi, verificando come cambia $J_{\text{VAL}}(\vartheta^i)$ in funzione di $\lambda^i$:
\[
\left[
\begin{array}{c}
\lambda^{(0)}\\
\lambda^{(1)}\\
\lambda^{(2)}\\
\lambda^{(3)}\\
\vdots\\
\lambda^{(k)}
\end{array}
\right]
\;\Rightarrow\;
\min_{\vartheta} J(\vartheta)
\;\Rightarrow\;
\left[
\begin{array}{c}
\vartheta^{(0)}\\
\vartheta^{(1)}\\
\vdots \\
\vartheta^{(k)}
\end{array}
\right]
\quad
\left[
\begin{array}{c}
J_{\text{VAL}}(\vartheta^{(0)})\\
J_{\text{VAL}}(\vartheta^{(1)})\\
\vdots \\
J_{\text{VAL}}(\vartheta^{(k)})
\end{array}
\right]
\;\Rightarrow\;
\underbrace{\text{BEST } J_{\text{VAL}}}_{\text{min}}
\;\Rightarrow\;
J_{\text{TEST}}\!\big(\vartheta^{\text{BEST}}\big)
\]


In questo modo si riesce a trovare il miglior parametro di regolarizzazione che minimizza l'errore sul validation set, e si può usare questo parametro per valutare il modello sul test set.

\paragraph{Altre curve.}
Un'altra curva potrebbe essere quella che mostra l'\textbf{errore} su \emph{train set} e \emph{validation set} in funzione del numero di campioni di training (figura \ref{fig:error_vs_training_size}). In questo modo si può capire se il modello soffre di overfitting o underfitting:

\begin{figure}[htbp]
	\centering
	\includegraphics[width=0.7\textwidth]{images/learning_curve_train_validation.png}
	\caption{Learning curves: errore su training \(J_{\mathrm{TRAIN}}\) e su validation \(J_{\mathrm{VAL}}\) in funzione del numero di esempi \(m\). \(J_{\mathrm{VAL}}\) decresce, \(J_{\mathrm{TRAIN}}\) cresce leggermente e il \emph{generalization gap} (freccia) si riduce; le curve si avvicinano all'errore irreducibile.}
	\label{fig:error_vs_training_size}
\end{figure}

La figura conferma il fatto che esiste il trade-off bias-varianza, in quanto:
\begin{itemize}
	\item Se siamo in presenza di \emph{high bias}, quindi un alto gap tra la curva di train e l'asse delle ascisse, aumentare il numero di dati di training non aiuta molto. 
	\item Se siamo in presenza di \emph{high variance}, quindi un grande gap tra l'errore commesso tra il training set e il validation set, aumentare il numero di dati di training potrebbe aiutare a migliorare i risultati (non sempre).
\end{itemize}


\backmatter
% (eventuali appendici)
% \appendix
% \include{appendici}

% (eventuale bibliografia)
% \bibliographystyle{plain}
% \bibliography{bibliografia}

\end{document}